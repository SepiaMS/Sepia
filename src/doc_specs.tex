%******************************************************************************%
%                                                                              %
%                                                         :::      ::::::::    %
%    doc_specs.tex                                      :+:      :+:    :+:    %
%                                                     +:+ +:+         +:+      %
%    By: aguellil <aguellil@student.42.fr>          +#+  +:+       +#+         %
%                                                 +#+#+#+#+#+   +#+            %
%    Created: 2016/03/24 20:44:16 by aguellil          #+#    #+#              %
%    Updated: 2016/03/25 15:47:58 by aguellil         ###   ########.fr        %
%                                                                              %
%******************************************************************************%

\documentclass[a4paper,12pt,francais]{article}
\usepackage[utf8]{inputenc}
\usepackage[T1]{fontenc}
\usepackage{babel}
\usepackage{hyperref}
\usepackage{xspace}

% Common abbreviations in our document
\newcommand{\SEP}{\textsc{sep}\xspace}
\newcommand{\Sepia}{\textsc{Sepia}\xspace}

% Change some existing settings
\renewcommand{\baselinestretch}{1.15}

% Title page
\title{\Sepia\\Cahier des charges}
\author{Ahmed \textsc{Guellil}\\
    Dimitri \textsc{Torterat}\\
    Pierre \textsc{Varin}
}
\date{\today} % Change to fixed date when this document reaches final version

%%%%%

\begin{document}
% Changes the items symbols of a list
\renewcommand{\labelitemi}{-}
\renewcommand{\labelitemii}{\tiny{+}}
% Modifies the table of contents title
\renewcommand{\contentsname}{Sommaire}

\maketitle

Ce document dresse le cahier des charges du projet \Sepia, un jeu vidéo mobile destiné à luter contre l’isolement des patients atteints de sclérose en plaques (\SEP). Ce RPG (jeu de rôle) permettrait de surveiller certains symptômes de manière non intrusive, tout en étant capable d’envoyer des messages automatiques aux proches ou médecins suivant les points obtenus, si le patient le souhaite.

% PS: a specs document SHOULD ONLY describe what the final result would look like; what it can or cannot do. It SHOULD NOT make assumptions or order as to which language, libraries, etc. to use.

\newpage
\tableofcontents
\newpage

\section{Problématique}
\subsection{Contexte}

La \emph{sclérose en plaques} (\SEP{}) est une maladie auto-immune, qui attaque le système nerveux et engendre de nombreux symptômes physiques et mentaux. Il n’existe pas de traitement curatif pour cette pathologie.

Indépendamment des signes neurologiques, les patients sont régulièrement atteints d’une fatigue intense, qui apparaît plus fréquemment et plus brutalement que la fatigue non liée à la maladie, et peut être véritablement invalidante pour la vie sociale, familiale et professionnelle.

Les statistiques de l’Inserm\footnote{Institut national de la santé et de la recherche médicale.} observent notamment que la maladie~:
\begin{itemize}
\item concerne environ 80~000 personnes en France (1 personne sur 1~000)~;
\item touche d’avantage de femmes (1 homme pour 3 femmes environ)~;
\item commence à montrer des symptômes sérieux vers les 30 ans.
\end{itemize}

De plus, l’Arsep\footnote{Fondation pour l'aide à la recherche sur la sclérose en plaques.} a constaté que 30~\% des enfants présentant une \SEP{} ont moins de 10 ans (l’âge moyen est de 12 ans).

\subsection{Symptômes}

%\subsubsection{La fatigabilité}
\subsubsection{Asthéniques}
Les patients atteints de \SEP{} peuvent être affectés par quatre types de fatigue~:
\begin{itemize}
    \item La fatigue «~normale~»~;
    \item L’asthénie liée aux états dépressifs~;
    \item La fatigue neuromusculaire~;
    \item La «~fatigue \SEP{}~», lassitude propre à la \SEP{}.
\end{itemize}

Cette dernière fatigue est généralisée, accablante, irrésistible, et peut apparaître à n’importe quel moment de la journée, et ce, sans avertissement. La personne peut même se sentir soudainement somnolente, voire tomber endormie.

\subsubsection{Moteurs}
\begin{itemize}
    \item Spasticité (résistance involontaire à un mouvement imposé)
    \item Syndrome vestibulaire~:
        \begin{itemize}
            \item Vertige rotatoire (tournis)
            \item Nystagmus (mouvements involontaires de l'oeil)
            \item Ataxie (manque de coordination fine des mouvements volontaires)
        \end{itemize}
    \item Syndrome cérébelleux~:
        \begin{itemize}
            \item Ataxie (station debout difficile, marche perturbée)
            \item Dysarthrie (trouble de l'articulation de la parole)
            \item Tremblements
        \end{itemize}
    \item Dysphagie (sensation de gêne ou de blocage ressentie au moment de l'alimentation)
    \item Douleur, anesthésie, voire paralysie faciale
    \item Spasmes
\end{itemize}
\subsubsection{Sensoriels}
\begin{itemize}
    \item Tactiles
        \begin{itemize}
            \item Hypoesthésie (diminution de la sensibilité de l'ensemble des fonctions sensorielles)
            \item Paresthésie (troubles de la sensibilité tactile~: fourmillements, picotements, engourdissements)
            \item Signe de Lhermitte (sensation de décharge électrique parcourant le rachis et les jambes lors de la flexion de la colonne cervicale)
        \end{itemize}
    \item Visuels
        \begin{itemize}
            \item Névrite optique rétro-bulbaire (baisse de l'acuité visuelle accompagnée de douleurs oculaires)
            \item Scotomes (tâches noires)
            \item Dyschromatopsie de l'axe rouge-vert (trouble de la vision des couleurs)
            \item Diplopie (image perçue dédoublée)
        \end{itemize}
    \item Auditifs
        \begin{itemize}
            \item Hyperacousie (intolérance aux sons variable en fréquence et en intensité)
        \end{itemize}
\end{itemize}
\subsubsection{Gastro-intestinaux/Uro-génitaux}
\begin{itemize}
    \item Rétention/Incontinence
    \item Diarrhée/Constipation
    \item Impuissance
\end{itemize}
\subsection{Objectifs} % Bienfaits et portée de notre jeu

À l’origine, l’application \Sepia{} se présente sous la forme d’un jeu vidéo afin de \emph{ne pas trop donner l’impression au patient qu’il utilise une application à portée médicale}. L’application devrait –~indirectement~– toucher l’ensemble de l’entourage de l’utilisateur~:
\begin{description}
    \item[Le patient lui-même] \Sepia{} contribuerait directement à son bien-être, en donnant des conseils sur –~par exemple~– l’hygiène de vie, la gestion du stress, le reconditionnement à l’effort.
    \item[Les proches] % FIXME
    \item[Le médecin] \Sepia{} ne se permettant pas d’agir en outil diagnostic, les statistiques (graphiques, etc.) recueillies par l’application permettraient au neurologue d’avoir une meilleure vision\footnote{Les rendez-vous chez un neurologue sont très rapides et espacés dans le temps~: un rendez-vous d’environ 15~minutes tous les 6~mois. Avoir une vision globale et plus ou moins exhaustive des évolutions pourrait aider le médecin.} sur l’apparition et l’évolution des symptômes.
\end{description}

\section{Description fonctionnelle}
% Introduire [une fois de plus] le principe et but du jeu.
% Inclure des schémas si besoin

L'application est centrée sur le patient, qu'elle considère comme le personnage principal du jeu de rôle, et se comporte alors comme une assistance digitale créant des interfaces entre l'univers virtuel et le monde réel dans lequel il évolue. Dès son lancement, elle propose l'accès à différents menus détaillés ci-après.

\subsection{Avatar}

L'avatar est la représentation virtuelle du patient. Il peut notamment reflèter sa condition par un système de \textit{sprites} ou de couleurs.

%\subsubsection{Création}
% Parler de la création du personnage et/ou du compte (si c’est important)
%\subsubsection{Évolution}

\subsection{Statut}

La page de statut présente des informations liées à l'évolution du patient.

\subsubsection{États}
Les états reflètent, en positif comme en négatif, la condition du patient (fatigue, paralysie, etc.). Ils sont représentés par des pictogrammes qui peuvent être activés ou désactivés de manière manuelle, voire semi-automatique si l'application est capable d'en reconnaître certains.

\subsubsection{Caractéristiques}
Les caractéristiques sont des données quantifiables permettant de juger de la progression du patient vis à vis des quêtes qu'il lui faut remplir. Elles sont augmentées lorsque le patient progresse dans la réalisation de ces quêtes, diminuées lorsqu'il est affecté de certains symptômes. Ces augmentations ou diminutions ont, en retour, une conséquence directe sur le niveau de difficulté des épreuves.

\subsection{Journal}
Le journal regroupe les quêtes disponibles, en cours, et passées du patient. Ces quêtes ont pour but d'aider à mieux surmonter sa maladie, notamment lors d'une poussée.

%\subsubsection{Quêtes réelles}
%\subsubsection{Quêtes virtuelles}
%\subsubsection{Quêtes sociales}

\subsubsection{Marche}
Le podomètre du téléphone nous donnera cette information, nous utiliserons cette donnée pour attribuer une monnaie virtuelle, permet acheter des objets qui seront placés dans l'inventaire.

\subsubsection{Mémoire}
Plusieurs types sont prévus.
\begin{itemize}
    \item Liste de mots à mémoriser à redonner.
    \item Mémoriser et redonner une séquence visuelle (exemple des cerlces de couleurs s'active les un après les autres).
\end{itemize}

\subsubsection{Vision}
\subsubsection{Équilibre}
L'utilisation du gyroscope du téléphone, une bille devra être au centre et maintenue en position durant un certaint temps.

\subsubsection{Fatigue}
L’état de fatigue du patient lui est directement demandée (méthode auto-évaluative), sur une échelle de 0 à 10.

\subsubsection{Sensations}
\subsubsection{Audition}
%\subsubsection{Mini-jeux}
% Parler des déplacements (et potentielle interaction avec podomètre, etc.)
\subsection{Inventaire}
L'inventaire contient des objets pouvant provenir directement de l'environnement réel du patient. Ces objets lui permettent de bénéficier d'aides à la progression.
%\subsection{Personnages Non Joueurs (PNJ)}
\end{document}
