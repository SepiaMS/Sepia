\documentclass[a4paper,12pt,francais]{article}
\usepackage[utf8]{inputenc}
\usepackage[T1]{fontenc}
\usepackage{babel}
\usepackage{hyperref}
\usepackage{xspace}

% Common abbreviations in our document
\newcommand{\SEP}{\textsc{sep}\xspace}
\newcommand{\Sepia}{\textsc{Sepia}\xspace}

% Change some existing settings
\renewcommand{\baselinestretch}{1.15}

% Title page
\title{\Sepia\\Cahier des charges}
\author{Ahmed \textsc{Guellil}\\
  Miguel \textsc{Soudan}\\
  Dimitri \textsc{Torterat}\\
  Pierre \textsc{Varin}
}
\date{\today} % Change to fixed date when this document reaches final version

%%%%%

\begin{document}
% Changes the items symbols of a list
\renewcommand{\labelitemii}{–}
% Modifies the table of contents title
\renewcommand{\contentsname}{Sommaire}

\maketitle

Ce document dresse le cahier des charges du projet \Sepia, un jeu vidéo mobile destiné à luter contre l’isolement des patients atteints de sclérose en plaques (\SEP). Ce RPG (jeu de rôle) permettrait de surveiller certains symptômes de manière non intrusive, tout en étant capable d’envoyer des messages automatiques aux proches ou médecins suivant les points obtenus, si le patient le souhaite.

% PS: a specs document SHOULD ONLY describe what the final result would look like; what it can or cannot do. It SHOULD NOT make assumptions or order as to which language, libraries, etc. to use.

\newpage
\tableofcontents
\newpage

\section{Problématique}
\subsection{Contexte}

La \emph{sclérose en plaques} (\SEP{}) est une maladie auto-immune, qui attaque le système nerveux et engendre de nombreux symptômes physiques et mentaux. Il n’existe pas de traitement curatif pour cette pathologie.

Indépendamment des signes neurologiques, les patients sont régulièrement atteints d’une fatigue intense, qui apparaît plus fréquemment et plus brutalement que la fatigue non liée à la maladie, et peut être véritablement invalidante pour la vie sociale, familiale et professionnelle.

Les statistiques de l’Inserm\footnote{Institut national de la santé et de la recherche médicale.} observent notamment que la maladie~:
\begin{itemize}
\item concerne environ 80~000 personnes en France (1 personne sur 1~000)~;
\item touche d’avantage de femmes (1 homme pour 3 femmes environ)~;
\item commence à montrer des symptômes sérieux vers les 30 ans.
\end{itemize}

De plus, l’Arsep\footnote{Fondation pour l'aide à la recherche sur la sclérose en plaques.} a constaté que 30~\% des enfants présentant une \SEP{} ont moins de 10 ans (l’âge moyen est de 12 ans).

\subsection{Symptômes}

%\subsubsection{La fatigabilité}
\subsubsection{Asthéniques}
Les patients atteints de \SEP{} peuvent être affectés par quatre types de fatigue~:
\begin{itemize}
    \item La fatigue «~normale~»~;
    \item L’asthénie liée aux états dépressifs~;
    \item La fatigue neuromusculaire~;
    \item La «~fatigue \SEP{}~», lassitude propre à la \SEP{}.
\end{itemize}

Cette dernière fatigue est généralisée, accablante, irrésistible, et peut apparaître à n’importe quel moment de la journée, et ce, sans avertissement. La personne peut même se sentir soudainement somnolente, voire tomber endormie.

\subsubsection{Moteurs}
\begin{itemize}
    \item Spasticité (résistance involontaire à un mouvement imposé)~;
    \item Syndrome vestibulaire~:
        \begin{itemize}
            \item Vertige rotatoire (tournis)~;
            \item Nystagmus (mouvements involontaires de l’œil)~;
            \item Ataxie (manque de coordination fine des mouvements volontaires).
        \end{itemize}
    \item Syndrome cérébelleux~:
        \begin{itemize}
            \item Ataxie (station debout difficile, marche perturbée)~;
            \item Dysarthrie (trouble de l’articulation de la parole)~;
            \item Tremblements.
        \end{itemize}
    \item Dysphagie (sensation de gêne ou de blocage ressentie au moment de l’alimentation)~;
    \item Douleur, anesthésie, voire paralysie faciale
    \item Spasmes
\end{itemize}
\subsubsection{Sensoriels}
\begin{itemize}
    \item Tactiles
        \begin{itemize}
            \item Hypoesthésie (diminution de la sensibilité de l'ensemble des fonctions sensorielles)~;
            \item Paresthésie (troubles de la sensibilité tactile~: fourmillements, picotements, engourdissements)~;
            \item Signe de Lhermitte (sensation de décharge électrique parcourant le rachis et les jambes lors de la flexion de la colonne cervicale).
        \end{itemize}
    \item Visuels
        \begin{itemize}
            \item Névrite optique rétro-bulbaire (baisse de l'acuité visuelle accompagnée de douleurs oculaires)~;
            \item Scotomes (tâches noires)~;
            \item Dyschromatopsie de l’axe rouge-vert (trouble de la vision des couleurs)~;
            \item Diplopie (image perçue dédoublée).
        \end{itemize}
    \item Auditifs
        \begin{itemize}
            \item Hyperacousie (intolérance aux sons variable en fréquence et en intensité).
        \end{itemize}
\end{itemize}
\subsubsection{Cognitifs}
\begin{itemize}
    \item Rétention/incontinence~;
    \item Diarrhée/constipation~;
    \item Impuissance.
\end{itemize}

\subsection{Objectifs} % Bienfaits et portée de notre jeu

À l’origine, l’application \Sepia{} se présente sous la forme d’un jeu vidéo afin de \emph{ne pas trop donner l’impression au patient qu’il utilise une application à portée médicale}. L’application devrait –~indirectement~– toucher l’ensemble de l’entourage de l’utilisateur~:
\begin{description}
    \item[Le patient lui-même] \Sepia{} contribuerait directement à son bien-être, en donnant des conseils sur –~par exemple~– l’hygiène de vie, la gestion du stress, le reconditionnement à l’effort.
    \item[Les proches] % FIXME
    \item[Le médecin] \Sepia{} ne se permettant pas d’agir en outil diagnostic, les statistiques (graphiques, etc.) recueillies par l’application permettraient au neurologue d’avoir une meilleure vision\footnote{Les rendez-vous chez un neurologue sont très rapides et espacés dans le temps~: un rendez-vous d’environ 15~minutes tous les 6~mois. Avoir une vision globale et plus ou moins exhaustive des évolutions pourrait aider le médecin.} sur l’apparition et l’évolution des symptômes.
\end{description}

\section{Description fonctionnelle}
% Introduire [une fois de plus] le principe et but du jeu.
% Inclure des schémas si besoin

L'application est centrée sur un avatar (symbolisant le patient), personnage principal d'un jeu de rôle, et lui propose plusieurs mini-jeux pour tester diverses aptitudes physiques ludiquement. L'application garde en mémoire ces statistiques pour avoir une trace de l'évolution des capacités du patient. Le patient-joueur est récompensé par une monnaie virtuelle, afin de l'encourager à continuer d'utiliser régulièrement l'application.

\subsection{Menu principal}

Le menu principal donne une vue rapide du compte~:

\subsubsection{Avatar}

L'avatar du patient est un personnage censé le symboliser.

\subsubsection{Niveau}

Élement hérité des jeux de rôle, le niveau est ici un \emph{gimmick} (artifice). Il représente concrètement l'ancienneté d'utilisation de l'application. À chaque multiple de cinq, le patient-joueur reçoit un objet en récompense.

Le niveau est obtenu lorsque la barre d'expérience est remplie. L'expérience est obtenue en complétant un mini-jeu, comme la monnaie virtuelle.

\subsubsection{Statut}

Le statut de l'avatar est altérable par le patient qui peut choisir dans une liste d'états divers.

Les états reflètent, en positif comme en négatif, la condition du patient (fatigue, paralysie, etc.). Ils sont représentés par des pictogrammes qui peuvent être activés ou désactivés de manière manuelle, voire semi-automatique si l'application est capable d'en reconnaître certains.

\subsubsection{Statistiques (aperçu)}

Les statistiques sont là aussi une mécanique de jeu héritée du jeu de rôle. Elles représente grossièrement des élements-clefs à surveiller chez le patient~: motricité, réflexes, mémoire, mobilité\dots

Elles évoluent selon les résultats des mini-jeux et peuvent être temporairement altérées par certains objets (voir la section \emph{Potions}). Ces augmentations ou diminutions ont, en retour, une conséquence directe sur le niveau de difficulté des épreuves.

Elles sont ici représentées sommairement. Pour une vision détaillée, voir la section \emph{Statistiques (détail)}.

\subsubsection{Monnaie virtuelle}

La monnaie virtuelle est obtenue par les mini-jeux et permet l'acquisition d'objets virtuels. Voir la section \emph{Magasin}.

\subsubsection{Icônes de menu}

Via des icônes est proposé l’accès rapide aux autres menus~:

\begin{itemize}
    \item Quêtes – Mini-jeux~;
    \item Inventaire~;
    \item Magasin~;
    \item Statistiques.
\end{itemize}

\subsection{Quêtes – Mini-jeux}

Ces mini-jeux nous permettent de visualiser et de surveiller l'évolution des capacités du patient. Chaque mini-jeu travaille sur une capacité précise illustrée par une statistique correspondante. Le résultat du mini-jeu influe sur l'évolution de la statistique et donne un montant de monnaie virtuelle variable.
L'utilisation des mini-jeux donne une fois par jour expérience et monnaie virtuelle calculées selon l'évolution des résultats.

\subsubsection{Mémoire}
Le joueur doit reproduire une combinaison de couleurs jouée préalablement par l'application. La longueur de la séquence, sa rapidité et le temps imparti pour la reproduction dépend de la difficulté.

\subsubsection{Motricité}
Le joueur doit garder une bille en équilibre en gardant son téléphone le plus immobile possible. La sensibilité de la bille dépend de la difficulté.

\subsubsection{Réflexes}
Le joueur est face à une version du jeu de la taupe~: il doit taper le plus vite possible sur des élements surgissant de trous en face de lui. Le rythme du jeu dépend de la difficulté.

\subsubsection{Mobilité}
Le joueur peut ici consulter le total de pas effectués dans la journée, comptabilisés par la fonction podomètre de l'application.

\subsection{Inventaire}
Le patient peut ici voir les objets qu'il a acquis soit par le bonus offert tous les niveaux multiple de cinq soit par achat au magasin.

\subsection{Magasin}

Le patient peut ici acheter grâce à la monnaie virtuelle divers objets.

\subsubsection{Couleurs}
Récompenses graphiques destinées à visuellement améliorer l'interface, partant du sépia pour graduellement incorporer les couleurs achetées.

\subsubsection{Potions}
Récompenses ludiques destinées (lors de l'utilisation) à modifier pour douze heures les statistiques pour influencer le niveau de difficulté des quêtes. À noter que la modification artificielle des statistiques n'influence pas le calcul des récompenses des mini-jeux, ni la courbe de progression des statistiques.

\subsubsection{Recettes}
Récompenses pratiques donnant au patient plusieurs recettes culinaires de plats équilibrés pour une meilleure hygiène alimentaire.
\emph{Section à développer}

\subsection{Statistiques (détail)}

Le patient peut ici voir l'évolution dans le temps des statistiques de son avatar, grâce à quatre courbes de couleurs différentes~:

\subsubsection{INT (Intelligence) – Mémoire}
Courbe de couleur bleue.

\subsubsection{STR (Strenght) – Motricité}
Courbe de couleur rouge.

\subsubsection{DEX (Dextérité) – Réflexes}
Courbe de couleur jaune.

\subsubsection{STA (Stamina) – Mobilité}
Courbe de couleur verte.

%\subsubsection{Création}
% Parler de la création du personnage et/ou du compte (si c’est important)
%\subsubsection{Évolution}

%\subsubsection{Quêtes réelles}
%\subsubsection{Quêtes virtuelles}
%\subsubsection{Quêtes sociales}

\subsubsection{Exercices physiques}
\begin{itemize}
	\item Marche\\
		Le podomètre du téléphone nous donnera cette information, nous utiliserons cette donnée pour attribuer une monnaie virtuelle, permet acheter des objets qui seront placés dans l'inventaire.
	\item Maintien de l'équilibre\\
		L'utilisation du gyroscope du téléphone, une bille devra être au centre et maintenue en position durant un certain temps.
\end{itemize}

\subsubsection{Activités mentales}
Plusieurs types sont prévus~:
\begin{itemize}
	\item Mémorisation
		\begin{itemize}
			\item Une liste d'items est à mémoriser dans un temps imparti puis est à restituer dans son intégralité.
			\item Une séquence visuelle apparait (par exemple des cercles de couleur s'activant les uns après les autres) et est à retracer par le joueur.
		\end{itemize}
	\item Résistance à la fatigue
		%L’état de fatigue du patient lui est directement demandée (méthode auto-évaluative), sur une échelle de 0 à 10.
\end{itemize}
\subsubsection{Développement social}
%\subsubsection{Mini-jeux}
% Parler des déplacements (et potentielle interaction avec podomètre, etc.)
\subsection{Inventaire}
L'inventaire contient des objets pouvant provenir directement de l'environnement réel du patient. Ces objets lui permettent de bénéficier d'aides à la progression.
%\subsection{Personnages Non Joueurs (PNJ)}
\fi
\end{document}
